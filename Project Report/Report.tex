\documentclass[11pt]{article}

\usepackage{fullpage}
\usepackage{graphicx}
\usepackage{amsmath}
\usepackage{amssymb}
\usepackage{amsthm}
\usepackage{fancyvrb}

\parindent0in
\pagestyle{plain}
\thispagestyle{plain}


%% UPDATE MACRO DEFINITIONS %%
\newcommand{\myname}{Nguyen Quang Hao Tran - c3409773}
\newcommand{\assignment}{Python Web Socket Assignment}
\newcommand{\duedate}{October 18, 2024}

%% DO NOT CHANGE ANYTHING BELOW THIS LINE %%

\begin{document}
	
	\textbf{University of Newcastle}\hfill\textbf{\myname}\\[0.01in]
	\textbf{ELEC3500 -- Telecommunication Network}\hfill\textbf{\assignment}\\[0.01in]
	\textbf{Prof.\ Duy Ngo}\hfill\textbf{\duedate}\\
	\smallskip\hrule\bigskip
	
	\section{Questions}
	\begin{enumerate}
		\item \textbf{Considering the functionality of this simple proxy server, identify and discuss at least three significant limitations of your proxy server. (6 Marks)}
			\begin{itemize}
				\item \textbf{The server does not handle any request other than GET requests}\\
				In this sever implementation, the sever accept GET request. The GET method is used to get data from the server. Those data could be webpage (html file), image, video, CSS file, JavaScript file, etc. The GET reqest does not change state of any resources on the server.\\
				The problem in this server in this server implementation is that the server does not andle other request such as POST, DELETE, PUT, etc.
					\begin{itemize}
						\item POST request: The POST method is used to send data to the server. The data in HTTP POST method is stored int he request body of the HTTP request.
						\item DELETE request: The DELETE method is used to remove a resource on the server.
						\item PUT request: The PUT method is to replace existing resources on the server
					\end{itemize}
				The absence of other request method will limit the functionality of the sever. The server will not be able to perform more advanced task such as submit username and password.\\
				
				\item \textbf{The sever does not handle different types of response other than '304 Not Modified'}\\
				In this proxy server implementation, the server only handles a '304 Not Modified' response to load cached data. It does not account for other responses such as '404 Not Found', '403 Forbidden', or similar errors. As a result, when an issue occurs, the server fails to provide informative feedback to the user about the nature of the problem. The user is left unaware whether the issue is caused by an internal server error or a client-side issue, leaving them unable to address the problem effectively if it is on their end.\\
				
				\item \textbf{The sever is single threaded, which significantly slow down sever response}\\
			\end{itemize}
		
		\item How could you implement potential improvements into this script to overcome these limitations? A flowchart or pseudocode may help to illustrate your answer. (6 marks)
		
		\item Is the proxy server using UDP or TCP sockets? How can you tell? What other protocols are involved?(2 marks)\\
		To know if the proxy server is using a UDP or TCP socket, I looked at the initialization of the socket server: \texttt{servSock = socket(AF\_INET, SOCK\_STREAM)}. The socket is using \texttt{SOCK\_STREAM} as the argument, which is used for a TCP connection. If it were to be UDP, the argument should be \texttt{SOCK\_DGRAM}.
		
		Other Protocols Involved:
		
			\begin{itemize}
				\item HTTP (Hypertext Transfer Protocol): The proxy server is fetching web pages, which are delivered using HTTP. The proxy forwards HTTP `GET` requests to the web server and receives HTTP responses.
				\item DNS (Domain Name System): DNS may be involved implicitly if the URL contains a domain name that needs to be resolved to an IP address before the proxy can communicate with the web server.
			\end{itemize}
		So, the proxy server uses TCP sockets, and the protocols involved are TCP, HTTP, and possibly DNS.
		
		\item In this question, you will put together much of what you have learned about Internet protocols. Suppose you buy a brand new computer, connect it to Ethernet, and want to download a Web page. What are all the protocol steps that take place, starting from powering on your PC to getting the Web page? Assume there is nothing in our DNS or browser caches when you power on your PC. (Hint: the steps include the use of Ethernet, DHCP, ARP, DNS, TCP, and HTTP protocols.) Explicitly indicate in your steps how you obtain the IP and MAC addresses of a gateway router. (6 marks)\\
		
		The first step after starting the computer is to use the Ethernet to use Dynamic Host Configuration Protocol (DHCP), a network management protocol, to configure devices on the Internet Protocol (IP) network. DHCP will provide the device with a unique IP address and also help us to get all neighbor routers' IP. Next, The computer will use Address Resolution Protocol (ARP) to map an IP address to a device's Media Access Control (MAC) address within a local network. The Domain Name System (DNS) is also obtained in this stage. When the user wants to fetch a webpage, the DNS will be used to find the IP address of the web page. Finally, the computer will send a Hypertext Transfer Protocol (HTTP) request to open the page in a browser. The server will then respond with the content and display it in the browser. The Transmission Control Protocol (TCP), IP datagram, and Ethernet frame are the result of the division and encapsulation of this request. The packet then goes through all interfaces to get to the destination.\\
	\end{enumerate}
\end{document}  